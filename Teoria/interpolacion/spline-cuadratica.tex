Es una técnica para crear una función suave que pase a través de un conjunto 
  de puntos dados. Muy diferente a la interpolación lineal, que conecta los puntos 
  con segmentos de línea recta. En éste caso se usa polinomios (grado 2) para crear
  dicha curva \textbf{suave}.\par
    \subsection{Concepto Básico}
    La idea es dividir el rango de datos en segmentos y, para cada segmento, ajustar 
    un polinomio cuadrático (de segundo grado). Estos polinomios están construidos de 
    manera que sean continuos en el punto de unión, es decir, la curva es suave en todos 
    los puntos de datos.
    \subsection{Pasos para Implementar la Interpolación Spline Cuadrática}
     \begin{enumerate}
        \item Dividir los puntos de datos en segmentos. Si tienes $n$ puntos, tendrás $n-1$ segmentos.
        \item Ajustar un polinomio cuadrático a cada segmento. La forma general de un polinomio cuadrático es:
        \[S_i(x) = a_i (x - x_i)^2 + b_i (x - x_i) + c_i\]
        donde $a_i$, $b_i$, y $c_i$ son los coeficientes del polinomio para el segmento $i$, y $x_i$ es el punto 
        de inicio del segmento.
        \item Imponer condiciones de continuidad: Los polinomios deben coincidir en el valor y en la primera 
        derivada en los puntos de unión.
      \end{enumerate}