
La interpolación es una técnica utilizada para estimar valores intermedios 
entre puntos de datos conocidos. A continuación, se describen varios métodos 
de interpolación, cada uno con sus características y aplicaciones:
\subsection{Tipos}
\subsubsection{Interpolación Lineal (\texttt{linear})}
Este es el método más simple, que conecta dos puntos adyacentes con una 
línea recta. Es adecuado para datos que siguen una tendencia lineal y 
proporciona una transición suave entre puntos.

\subsubsection{Interpolación Más Cercana (\texttt{nearest})}
Este método asigna al punto interpolado el valor del punto de datos más
 cercano. Es útil cuando se requiere que los valores interpolados sean 
 iguales a los puntos de datos originales, sin generar valores intermedios.

\subsubsection{Interpolación Más Cercana hacia Arriba (\texttt{nearest-up})}
Similar a \texttt{nearest}, pero siempre selecciona el siguiente valor más 
cercano hacia arriba. Es útil cuando se desea evitar subestimaciones.

\subsubsection{Interpolación por Zonas (\texttt{zero})}
Este método mantiene constantes los valores de la interpolación hasta el 
siguiente punto de datos, lo que produce una función por partes. Es útil 
en sistemas donde los valores cambian bruscamente.

\subsubsection{Interpolación Lineal Segmentada (\texttt{slinear})}
Una versión mejorada de la interpolación lineal, que utiliza segmentos 
lineales entre puntos de datos. Es más suave que la interpolación lineal 
simple, especialmente en datos no lineales.

\subsubsection{Interpolación Cuadrática (\texttt{quadratic})}
Utiliza polinomios de segundo grado para interpolar entre puntos. 
Es más preciso que los métodos lineales en datos con curvaturas suaves y no lineales.

\subsubsection{Interpolación Cúbica (\texttt{cubic})}
Este método utiliza polinomios de tercer grado y es altamente adecuado 
para datos que requieren una curva suave y continua. Proporciona una mejor 
aproximación que los métodos lineales y cuadráticos.

\subsubsection{Interpolación Previa (\texttt{previous})}
Asigna al punto interpolado el valor del punto de datos anterior. Es útil en 
situaciones donde los valores deben permanecer constantes hasta el próximo 
cambio de datos.

\subsection{Comparación de Métodos}
Cada método de interpolación tiene aplicaciones específicas dependiendo de la 
naturaleza del conjunto de datos y la suavidad deseada en la curva de interpolación. 
Los métodos lineales son simples y rápidos, pero pueden ser inexactos en datos 
con cambios no lineales. Los métodos cuadráticos y cúbicos son más complejos, 
pero ofrecen una mayor precisión y suavidad en la interpolación.