La función \textbf{splev} es parte del módulo \texttt{scipy.interpolate} y se 
utiliza para evaluar un spline en puntos específicos. Después de haber ajustado 
un spline usando \texttt{splprep}, puedes usar \texttt{splev} para calcular el 
valor del spline en cualquier conjunto de parámetros.

\subsection*{¿Cómo funciona \texttt{splev}?}

\textbf{splev} toma un conjunto de valores de parámetros y un objeto spline 
(generalmente el resultado de \texttt{splprep}) y devuelve los valores 
correspondientes del spline en esos puntos.

\subsection*{Parámetros importantes}

\begin{itemize}
    \item \texttt{u}: Un array de valores de parámetros en los cuales deseas 
    evaluar el spline. Estos valores deben estar en el rango \texttt{[0, 1]} 
    si no has especificado otro rango en \texttt{splprep}.
    \item \texttt{tck}: El objeto spline, que es el resultado de la función 
    \texttt{splprep}. Contiene la información sobre el spline, incluyendo el 
    vector de nodos, los coeficientes y el grado del spline.
\end{itemize}

\subsection*{Resultado}

\begin{itemize}
    \item \texttt{values}: Un array de valores que corresponde a la evaluación 
    del spline en los puntos especificados por \texttt{u}. Si estás trabajando 
    en un espacio multidimensional, este será una lista de arrays, uno para cada 
    dimensión.
\end{itemize}
